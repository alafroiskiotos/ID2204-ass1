\documentclass[a4paper, 11pt]{article}

\usepackage{mathtools}

\newcommand{\tab}[1]{\hspace{.2\textwidth}\rlap{#1}}

\begin{document}

\title{Assignment 1\\ID2204 -- Constraint Programming}

\author{Antonios Kouzoupis $<$890121-8837$>$ -- Lorenzo Corneo $<$890225-1290$>$\\\{antkou,corneo\}$@$kth.se}

\maketitle

\section{Sudoku}

For the Sudoku puzzle, initially we had to add a custom option
handler. We did this to parameterize which puzzle from the examples
would solve. The Script takes the command line options as argument and
parses the puzzle to solve and the Integer Consistency Level. It
also initializes a \emph{IntVarArray} matrix with size 81 (9 * 9) and
values from 1 to 9.

After that, it creates an integer array by flattening the sudoku
example matrix. The Sudoku grid is represented by the
\emph{IntVarArray}, created before, and transformed into a Matrix
object. The constraints we put are integer equality for every element
in the sudoku example that is different from zero. Also, the numbers
in each row, column and every 3x3 sub-matrix should be distinct.

For branching there was no strategy that outperformed an other one. We
experimented with different parameters and we concluded that in most
of the examples the strategy that performed better is
INT\_VAR\_MIN\_MAX for variable selection and INT\_VAL\_MED for value
selection.

Similarly, there is no best option for the Consistency Level. The
Domain propagation performed better than other levels in most of the
given sudoku puzzles. But for instance, the Domain propagation for example
4 results in 4 node traversed and 1 failures whereas the Default level
results in 23 nodes traversed and 10 failures. The Bounds
propagation performs better than the default one and the Domain
propagation by achieving 2 nodes
traversed and 0 failures. On puzzle number 17 all Consistency levels
traverse the same number of nodes and perform the same number of
failures.

\section{n-Queens}

\subsection{Constraints}

The constraints that have to be applied to our model are essentially
3: row, column and diagonal constraints. In a matrix modelization of
the chessboard for the n-queens problem, the management of the
diagonal constraints resulted quite tricky. In fact, for each generic
cell of the matrix we have to define a constraint.

Nevertheless, an optimization to reduce the number of constraints for
the diagonal can be made if we propagate these constraints starting
from the edges of the matrix (3 are enough). This because, if we
defined a diagonal constraint for each cell we would obtain
overlapping! For example, in a 4x4 matrix we define only main diagonal
constraint for cells [0,0], [0,1], [0,2], [1,0], [2,0] while for the
secondary diagonal constraints [0,1], [0,2], [0,3], [1,3], [2,3].

All the types of constraints are defined by the method
\texttt{linear(*this, x, IRT\_LQ, 1)} because each constraint has to
be less equal than one (this guarantees only correct placements of
queens).

\subsection{Branching}

We applied two strategies for the branching method: INT\_VAL\_MAX and
IN\_VAL\_MIN. In the former case, the brancher finds firstly the
solution that have as many as possible MAX\_VAL (in our case 1) in the
beginning of the matrix (first elements of the array). As a
consequence we expect to find a 1 in the first position (if there is
such a solution). In the latter case, we expect exactly the opposite
so that we will find in the firsts positions the value zero.

\subsection{Pros and cons}

One of our favourite pros, but less relevant, is that the matrix is
identical to the chessboard and for that it is immediate to
understand. In the method presented during the lectures we had to put
some more effort to figure out the representation of the solution.

In addition, on this model the definition of the constraints is less
complicated because the matrix is the same model of the
chessboard. Also this one is not a pros concerning performance or
computation, but nice if you have to develop the solution.

Furthermore, in the given example the number of propagators is
$3n^{2}$ while in our solution is $7n-8$.

The only cons we have found is that the matrix uses more memory than
the simple array. This result in a less efficient model from the point
of view of the resources used.

\section{Compile and Run}

To compile the two programs type \texttt{make} in the main
directory. To delete the object files type \texttt{make clean} and to
delete both executable and object type \texttt{make dist-clean}.
	
In order to run the sudoku program, type \texttt{./sudoku [-puzzle
  NUM]} where NUM is the number of the sudoku puzzle in the given
examples file. Default puzzle is the first one. Also it supports every
option of the Options class.

To execute the n-Queens program, type \texttt{./matrix\_queens [SIZE]}
where SIZE is the N of a N*N matrix. By default, it executes for N =
4. To view all the solutions found add the \texttt{-solutions 0} option.
\end{document}
