\documentclass[a4paper, 11pt]{article}

\newcommand{\tab}[1]{\hspace{.2\textwidth}\rlap{#1}}

\begin{document}

\title{Assignment 1\\ID2204 -- Constraint Programming}

\author{Antonios Kouzoupis $<$890121-8837$>$ -- Lorenzo Corneo $<$890225-1290$>$\\\{antkou,corneo\}$@$kth.se}

\maketitle

\section{Sudoku}

For the Sudoku puzzle, initially we had to add a custom option
handler. We did this to parameterize which puzzle from the examples
would solve. The Script takes the command line options as argument and
parses the puzzle to solve and the Integer Consistency Level. It
also initializes a \emph{IntVarArray} matrix with size 81 (9 * 9) and
values from 1 to 9.

After that, it creates an integer array by flattening the sudoku
example matrix. The Sudoku grid is represented by the
\emph{IntVarArray}, created before, and transformed into a Matrix
object. The constraints we put are integer equality for every element
in the sudoku example that is different from zero. Also, the numbers
in each row, column and every 3x3 sub-matrix should be distinct.

For branching there was no strategy that outperformed an other one. We
experimented with different parameters and we concluded that in most
of the examples the strategy that performed better is
INT\_VAR\_MIN\_MAX for variable selection and INT\_VAL\_MED for value
selection.

Similarly, there is no best option for the Consistency Level. The
Domain propagation performed better than other levels in most of the
given sudoku puzzles. But for instance, the Domain propagation for example
4 results in 4 node traversed and 1 failures whereas the Default level
results in 23 nodes traversed and 10 failures. The Bounds
propagation performs better than the default one and the Domain
propagation by achieving 2 nodes
traversed and 0 failures. On puzzle number 17 all Consistency levels
traverse the same number of nodes and perform the same number of
failures.

\section{n-Queens}

\section{Compile and Run}

To compile the two programs type \texttt{make} in the main
directory. To delete the object files type \texttt{make clean} and to
delete both executable and object type \texttt{make dist-clean}.
	
In order to run the sudoku program, type \texttt{./sudoku [-puzzle
  NUM]} where NUM is the number of the sudoku puzzle in the given
examples file. Default puzzle is the first one. Also it supports every
option of the Options class.

\end{document}
